\subsection{Gaussian Splatting}
Przy pomocy biblioteki \textit{gsplat}\cite{ye2024gsplatopensourcelibrarygaussian} zawierającej implementację \textit{Gaussian Splatting} w Pythonie wykonaliśmy eksperymenty polegające na uruchomeniu algorytmu dla różnych wartości hiperparametrów w celu znalezienia wartości, które prowadzą do jak najbardziej optymalnego procesu trenowania w kontekście czasu trwania i wykorzystania pamięci. 

Na wejściu algorytmu podawana jest otrzymywana w procesie rekonstrukcji chmura punktów, która jest bazą do dalszego dzielenia i powstawania "gaussianów", a ich parametry: pozycja, kolor, skala i rotacja są optymalizowane przy pomocy metody spadku wzdłuż gradientu. Metryki przyjęte do oceny jakości to SSIM (Structural Similarity Index Measure), PSNR (Peak Signal-to-Noise Ratio) oraz LPIPS (Learned Perceptual Image Patch Similarity).

W wyniku przeprowadzenia eksperymentów okazało się, że najważniejszymi sterującymi procesem parametrami są 
\begin{enumerate}
    \item Liczba Gaussianów: w przypadku scen urbanistycznych w celu oddania odpowieniej szczegółowości potrzebne jest parę milionów Gaussianów, dla naszych scen było to zwykle 3 mln.
    \item Strategia i częstość adaptacji: określają w jaki sposób oraz jak często dodawane i usuwane są Gaussiany. 
    \item Liczba iteracji: zwykle im dłużej trenowana jest scena tym lepsze wyniki otrzymujemy, jednak zależy to również od przyjętej strategii. Liczba ta wpływa bezpośrednio na czas trenowania, powinna wynieść nie mniej niż paręnaście tysięcy.
    \item Stopień zmiennych harmonicznych: wyrażają one kolor, im większy stopień tym lepsza jakość sceny, ale też zwiększone zużycie pamięci i wydłużony czas trenowania. 
\end{enumerate}

Poniżej przedstawione są przykładowe wizualizacje. Renderowania zostały wykonane przy pomocy biblioteki nerfview która również służy do wizualizacji splatów. Na poniższych rysunkach są od lewej do prawej: prawdziwe zdjęcie i widok modelu.

\begin{figure}[!h]
    \centering
    \includegraphics[width=1.0\linewidth]{images/sks_viper_0008.png}
    \caption{Scena SKS}
    \label{fig:sks_gs}
\end{figure}

\begin{figure}[!h]
    \centering
    \includegraphics[width=1.0\linewidth]{images/c5_mouse_0001.png}
    \caption{Scena C5}
    \label{fig:c5_gs}
\end{figure}

\begin{figure}[!h]
    \centering
    \includegraphics[width=1.0\linewidth]{images/c7_gepard_0006.png}
    \caption{Scena C7}
    \label{fig:c7_gs}
\end{figure}

\begin{table}[!h]
    \centering
    \begin{tabular}{|c|c|c|c|c|}
    \hline
    scena & PSNR & SSIM & LPIPS & liczba gaussianów \\
    \hline 
    SKS & 22.03 & 0.71 & 0.25 & 2,937,549 \\
    \hline 
    C5 & 0.0 & 0.0 & 0.0 & 0.0 \\
    \hline 
    C7 & 22.63 & 0.72 & 0.29 & 3,000,000 \\
    \hline
    \end{tabular}
\caption{Całościowe metryki dla testowych scen}
\label{table:tab_conf_sks}
\end{table}