\subsection{Segmentacja}
Używając biblioteki \textit{PyTorch} do uczenia głębokiego, w oparciu o istniejące rozwiązania i aktualny stan wiedzy, przygotowaliśmy i wytrenowaliśmy własne modele segmentacji semantycznej na chmurze punktów, otrzymanej w naszej aplikacji w wyniku działania poprzednich etapów i przekształceń na danych z nimi związanych. 
\textbf{Segmentacja semantyczna}
Zadanie postawione przed modelem jest jednym z kategorii zadań widzenia komputerowego. Polega ono na przypisaniu każdemu z punktów w chmurze etykiety określającej do jakiego rodzaju obiektu on przynależy, na przykład: czy jest on częścią budynku, drogi, samochodu, czy terenu zielonego. Obiektem wejścia jest zatem chmura punktów, natomiast na wyjściu otrzymujemy 