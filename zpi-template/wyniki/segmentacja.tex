\subsection{Segmentacja semantyczna}
Otrzymana w wyniku poprzednich etapów chmura punktów poddawana procesowi segmentacji semantycznej, czyli przypisaniu każdemu z punktów odpowiedniej kategorii semantycznej opisującej obiekt, w skład którego wchodzi. Do wybranych (na podstawie popularnych w literaturze zbiorów danych służących za punkt odniesienia w testowaniu modeli) kategorii semantycznych należą między innymi \textit{budynek}, \textit{droga}, czy też \textit{zieleń miejska}. 
Używając biblioteki \textit{PyTorch} do uczenia głębokiego, w oparciu o istniejące rozwiązania i aktualny stan wiedzy, przygotowano i wytrenowano własne modele \emph{sieci neuronowych} do segmentacji semantycznej.
W procesie eksperymentowania z różnymi architekturami i sposobami implementacji procesu treningu i predykcji za kluczowe pod względem wpływu na osiągi otrzymanego modelu należy uznać:
\begin{enumerate}
    \item próbkowanie - przy przetwarzaniu zbiorów danych, w przypadku których określenie porządku jest z punktu widzenia efektywności rozwiązania bezcelowe, a które jednocześnie z punktu widzenia modelu mogą osiągać różne rozmiary, ważnym elementem procesu zarówno treningu, jak i predykcji jest odpowiednie próbkowanie całego zbioru. Jest to niezbędne ze względu na architekturę sieci neuronowych, która zakłada stały rozmiar wejścia do modelu. W obrębie tego problemu należy wyróżnić następujące czynniki:
    \begin{enumerate}
        \item rozmiar próbki - zbyt mały może uniemożliwić uchwycenie zależności pomiędzy zbliżonymi do siebie w chmurze punktami,
        \item sposób próbkowania - wpływa na zależność uchwyconych w wyniku procesu uczenia wzorców od bardziej lub mniej odległych od siebie punktów. Może prowadzić do swego rodzaju zdominowania segmentowanych punktów przez jedną kategorię.
    \end{enumerate}
    \item niezbalansowany zbiór danych - w przypadku rozpatrywania dużych scen miejskich naturalnym jest pojawienie się mniej (\textit{samochody}, \textit{tory kolejowe}) i bardziej (\textit{budynki}) popularnych kategorii semantycznych. Jest to klasyczny problem uczenia maszynowego na niezbalansowanym zbiorze danych, który, niezaadresowany, prowadzi do dominacji zbioru przez punkty popularniejszych kategorii, w efekcie przekładając się na słabszą generalizację otrzymanego modelu, w szczególności dla mniej popularnych kategorii semantycznych. W toku prac rozważano dwa sposoby radzenia sobie z tym problemem:
    \begin{enumerate}
        \item próbkowanie - opisane wyżej,
        \item ważenie funkcji straty - \textit{karze} model za omijanie mniej popularnych kategorii semantycznych. Technika ta może prowadzić do nadreprezentacji tych kategorii w otrzymanym w wyniku predykcji zbiorze, jednakże przy odpowiedniej implementacji nieco słabsze wartości metryk dla popularnych kategorii są \textit{nomen omen} balansowane przez lepsze wyniki na niedoreprezentowanych w zbiorze treningowym kategorii, prowadząc w efekcie do lepszych wartości metryk dla całego zbioru testowego.
    \end{enumerate}
    \item dobór danych treningowych, walidacyjnych i testowych - typowy problem dla uczenia maszynowego. Dbając o generalizację modelu nie możemy dopuścić do \textit{wycieku danych}, tj. sytuacji, w której obiecujące wyniki są spowodowane nie ową generalizacją, a pewnego rodzaju pokrewieństwem danych służących do treningu modelu i jego oceny w trakcie tego procesu lub po jego zakończeniu. Rozwązaniem oprócz odpowiedniego podziału i doboru danych jest użycie technik przetwarzania chmur punktów związanych np. z ich obracaniem lub skalowaniem, tak aby wyuczone wzorce podlegały jak najlepszemu uogólnieniu.
\end{enumerate}

\begin{table}[!h]
    \centering
    \begin{tabular}{|c|c|c|c|c|c|c|}
    \hline
    scena & PSNR & SSIM & LPIPS & liczba gaussianów & czas trenowania & pamięć pliku (MB) \\
    \hline 
    SKS & 22.03 & 0.71 & 0.25 & 2,937,549 & 2h53m & 661 \\
    \hline 
    C5 & 21.98 & 0.71 & 0.26 & 4,564,464 & 10h40m & 675 \\
    \hline 
    C7 & 22.63 & 0.72 & 0.29 & 3,000,000 & 15h15m & 675 \\
    \hline
    \end{tabular}
\caption{Całościowe metryki dla testowych scen. Otrzymane wartości PSNR, SSIM oraz LPIPS zwykle świadczą o dobrej jakości scenie, która oddaje wystarczające szczegóły i wygładzone artefakty.}
\label{table:tab_seg_met}
\end{table}

Otrzymany w wyniku tego procesu eksperymentowania model wdrożono w celu jego używania na \textit{niewidzianych} przez niego dotychczas danych.