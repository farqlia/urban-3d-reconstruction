\subsection{Wizualizacja}
Wizualizacja modeli 3D stanowi wyzwanie dla użytkownika końcowego,
głównie z powodu braku spójnych platform umożliwiających realizację
całego procesu – od wgrania plików wejściowych po interakcję z modelem – w ramach jednej aplikacji.
Dostępne na rynku rozwiązania wymagają korzystania z aplikacji trzecich i pewnej wiedzy technicznej,
co prowadzi do problemów z integracją i spójnością działania.

Celem projektu było stworzenie intuicyjnego, dynamicznego i responsywnego interfejsu zintegrowanego z wydajnym renderingiem GPU.
Aplikacja umożliwia użytkownikowi końcowemu realizację pełnego procesu
wizualizacji – od tworzenia i modyfikacji modelu po jego segmentację i wyświetlanie – w jednej aplikacji.
Projekt rozwiązuje problem fragmentarycznej funkcjonalności dostępnych aplikacji, oferując spójne środowisko do obsługi modeli 3D.

Korzyści z realizacji projektu obejmują:

\begin{itemize}
    \item zwiększoną wydajność dzięki GPU,
    \item eliminację konieczności korzystania z wielu narzędzi,
    \item uproszczony proces użytkowania, co zwiększa dostępność aplikacji dla mniej zaawansowanych użytkowników.
\end{itemize}