\begin{abstract}
    Przedmiotem projektu było wykonanie aplikacji wykorzystującej metody fotogrametrii do modelowania trójwymiarowych scen miejskich. Zaimplementowany program umożliwia użytkownikowi przy użyciu interfejsu graficznego zrekonstruowanie chmury punktów na podstawie podanych na wejściu zdjęć fotogrametrycznych, za pomocą dostosowanych do specyfiki problemu w toku eksperymentów rozwiązaniach \textit{structure from motion} i \textit{gaussian splattingu}. Otrzymana w ten sposób chmura może być przez użytkownika poddana segmentacji semantycznej, przeprowadzanej przez wytrenowany do tego celu model sztucznej inteligencji w postaci sieci neuronowej. Aplikacja umożliwia również wizualizację wykonanych obliczeń, która możliwa jest dzięki użyciu przystosowanego dla większej wydajności mechanizmu renderowania.

    Wytworzony produkt informatyczny, ze względu na integrację wielu rozwiązań i efektywną implementację procesu \textit{end-to-end}, przejawia potencjał w zastosowaniach biznesowych począwszy od branż takich jak gry wideo, przez architekturę, robotykę, pojazdy autonomiczne, skończywszy na modelowaniu urbanistycznym.
\end{abstract}
