\begin{abstract}
    Celem pracy jest wykonanie aplikacji, która wykorzystuje metody fotogrametrii do modelowania miejskich scen 3D. Dane wejściowe stanowią zdjęcia obszarów miejskich, które są przetwarzane w celu stworzenia modelu 3D, a następnie segmentowane na obiekty przestrzeni miejskiej, takie jak budynki, tereny zielone, itp.. Aplikacja będzie wizualizować model oraz wyniki segmentacji semantycznej.

    Innowacyjność tego projektu polega na połączeniu, adaptacji i udoskonaleniu najlepszych dostępnych rozwiązań, takich jak Gaussian Splatting i PointNet, aby stworzyć nowy, kompleksowy produkt.

    Przetwarzanie dużych scen miejskich jest wyzwaniem dla obecnie istniejących rozwiązań, które skupiają się głównie na pojedynczych obiektach lub zamkniętych scenach. Typowa scena miejska natomiast może obejmować setki zdjęć, a powstała chmura może zawierać miliony punktów. Dodatkowym wyzwaniem jest niezbalansowana reprezentacja kategorii semantycznych. Nasze rozwiązanie ma na celu efektywne przetwarzanie dużych zbiorów danych przy rozsądnym zużyciu zasobów czasowych i pamięciowych.

    Zastosowania biznesowe otrzymywanych w ten sposób modeli 3D są szerokie: od gier wideo, przez architekturę, robotykę, pojazdy autonomiczne, po modelowanie urbanistyczne.
\end{abstract}
 
{\bf Keywords:} %\keywords
{Rekonstrukcja SFM}
