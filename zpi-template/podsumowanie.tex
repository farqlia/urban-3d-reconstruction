\section{Podsumowanie}
Rekonstrukcja i klasyfikacja krajobrazów urbanistycznych ma wiele potencjalnych zastosowań w dziedzinach takich jak \textit{Smart City} czy też \textit{Virtual Reality}. Projekt "urb3d" pokazał, że wykonanie takiego oprogramowania jest możliwe przy pomocy integracji istniejących rozwiązań i ich adaptacji.

Opracowane oprogramowanie umożliwia kompleksowe modelowania obszarów miejskich z wykorzystaniem fotogrametrii. Wśród oferowanych użytkownikowi funkcjonalności znajdują się m.in. dostosowywanie parametrów, wczytywanie zdjęć, uruchamianie poszczególnych etapów oraz przeglądanie rezultatów. Odpowiednio dobrane i dostrojone algorytmy zapewniają jakościowe wyniki, które mogą być wykorzystywane razem lub oddzielnie.

\subsection{Kierunki rozwoju}
W przyszłości projekt mógłby obejmować rekonstrukcję jeszcze większych obszarów urbanistycznych, co wiązałoby się z koniecznością zastosowania bardziej zaawansowanych technik optymalizacji. Warta wprowadzenia mogłaby okazać się kompresja danych w celu zmniejszenia końcowej liczby gaussianów w algorytmie \textit{splattingu}. Inny kierunek rozwoju to trenowanie modeli na różnych poziomach szczegółowości (ang. Level of Detail - LoD). Otwartym rozdziałem jest też testowanie kolejnych modeli segmentacji semantycznej i ich dostrajanie.

\subsection{Podziękowania}
Szczególne podziękowania należą się operatorce drona Paulinie, bez której z pewnością sukces w obszarze akwizycji danych nie byłby możliwy.