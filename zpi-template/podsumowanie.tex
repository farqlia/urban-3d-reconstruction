\section{Podsumowanie}
Rekonstrukcja i klasyfikacja krajobrazów urbanistycznych ma wiele potencjalnych zastosowań w dziedzinach takich jak \textit{Smart City} czy też \textit{Virtual Reality}. Projekt "urb3d" pokazał, że wykonanie takiego oprogramowania jest możliwe przy pomocy integracji istniejących rozwiązań i ich adaptacji. 

\subsection{Wnioski}
Zaprojektowane oprogramowanie zapewnia intuicyjne korzystanie z funkcjonalności takich jak dostosowywanie parametrów, wczytywanie zdjęć, uruchamianie poszczególnych etapów oraz przeglądanie rezultatów. Odpowiednio dobrane algorytmy zapewniają jakościowe wyniki, które mogą być dalej wykorzystane razem lub oddzielnie. 

\subsection{Kierunki rozwoju}
Większe obszary / Rekonstrukcji? 
W przypadku algorytmu Gaussian Splatting możliwe by było zastosowanie technik kompresji zmniejszających końcową liczbę gaussianów, lub trenowanie na różnych poziomach szczegółowości (ang. \textit{LoD}).
Segmentacja? 
Rendering? 

\subsection{Podziękowania}
??