\section{Podsumowanie}
Rekonstrukcja i klasyfikacja krajobrazów urbanistycznych ma wiele potencjalnych zastosowań w dziedzinach takich jak \textit{Smart City} czy też \textit{Virtual Reality}. Nasz projekt pokazał, że wykonanie takiego oprogramowania jest możliwe przy pomocy integracji istniejących rozwiązań i ich adaptacji. 

\subsection{Wnioski}
Kluczowymi czynnikami decydującymi o wykorzystaniu danego rozwiązania są: wykorzystanie zasobów, łatwość użycia oraz oferowane funkcje. Nasz projekt dzięki przejrzystemu interfejsowi użytkownika zapewnia intuicyjne korzystanie z funkcjonalności takich jak dostosowywanie parametrów, wczytywanie zdjęć, uruchamianie poszczególnych etapów oraz przeglądanie rezultatów. Dzięki trafnemu wyborowi algorytmów zapewniamy odpowiednią jakość wyników, które mogą być praktycznie zastosowane razem lub oddzielnie.  

\subsection{Kierunki rozwoju}
Większe obszary / Rekonstrukcji? 
Algorytm Gaussian Splatting - wiecej opcji / wariancji 
Segmentacja? 
Rendering? 

\subsection{Podziękowania}
?