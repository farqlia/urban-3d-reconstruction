\subsection{Stan wiedzy}

Unikalność naszego projektu wynika z połączenia wielu rozwiązań które istnieją samodzielnie na rynku. Alogrytm Structure-from-Motion jest popularną fotogrametryczną techniką otrzymywania chmury punktów ze zbioru zdjęć i jego implementacja oferowana jest m. in. przez oprogramowanie COLMAP. W przypadku modelu 3D można napotkać różne adaptacje algorytmu Gaussian Splatting, jak np. CityGaussian .... Ważnym krokiem jest również filtracja chmury punktów w celu usunięcia odstających punktów lub tych nieistotnych dla wyników klasyfikacji.  

\textbf{Wykorzystane oprogramowanie}
z fiszki można wrzucić 