\section{Stan wiedzy}

Unikalność naszego projektu wynika z połączenia wielu rozwiązań które istnieją samodzielnie na rynku. Alogrytm Structure-from-Motion jest popularną fotogrametryczną techniką otrzymywania chmury punktów ze zbioru zdjęć i jego implementacja oferowana jest m. in. przez oprogramowanie COLMAP. 

W przypadku modelu 3D często stosowaną techniką są siatki, jednak ich wadą jest niekompatybilność z algorytmami sztucznej inteligencji. Popularne są też rozwiązania wykorzystujące sieci neuronowe jak np. NeRF, ale długi czas trenowania jest nieefektywny. Z tego powodu zdecydowaliśmy się na wykorzystanie algorytmu Gaussian Splatting, który buduje model sceny z tzw. gaussianów, które można interpretować jako punkty rozmyte. Można napotkać różne adaptacje tego algorytmu, jak np. CityGaussian.

Ważnym krokiem jest również filtracja chmury punktów w celu usunięcia odstających punktów lub tych nieistotnych dla wyników klasyfikacji. W tym obszarze znajdują się np. techniki statystyczne czy oparte na sąsiedztwie. 

Istniejące architektury sieci neuronowych dla zadania segmentacji są głównie przeznaczone dla scen zamkniętych lub pojedynczych obiektów. Popularnym rozwiązaniem jest PointNet oraz jego następnik PointNet++ oparte na wielowarstwowym perceptronie, jak i również bardziej skomplikowane rozwiązania jak KPConv wykorzystujące konwolucje. Wyzwaniem dla naszego projektu będzie dostosowanie takich architektur do chmury punktów o wielkości rzędu milionów punktów.  

W przypadku renderowania istnieją rozwiązania przeznaczone zarówno do chmur punktów, jak i do splatów, jednak wyzwaniem jest ...

\textbf{Wykorzystane oprogramowanie}

Głównymi językami programowania są C/C++ oraz Python. 