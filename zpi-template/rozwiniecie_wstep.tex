\section{Wprowadzenie}
Nasz projekt skupia się na problemie rekonstrukcji trójwymiarowej scen urbanistycznych, jej klasyfikacji oraz wizualizacji. Warto podkreślić, że obszar naszej pracy jest relatywnie nowy i stawia wyzwania związane z efektywnością przetwarzania dużego zbioru danych - w naszym przypadku chmury punktów, która może składać się nawet z paru milionów punktów. Na rynku dostępne są rozwiązania które możemy wykorzystać, więc naszym głównym celem jest zbadanie ich użyteczności w naszym problemie i ich ewentualna adaptacja. 

Nasze rozwiązanie będzie umożliwiało przeprowadzenie rekonstrukcji do modelu trójwymiarowego na podstawie odpowiednio przygotowanego zbioru zdjęć, klasyfikację otrzymanej sceny na zbiór pre-definiowanych klas istotnych w kontekście scen urbanistycznych, oraz wizualizację wykonanych obliczeń. 

Jako zespół stawiamy następujące cele, które chcemy zrealizować:

\begin{enumerate}
    \item Skomponowanie własnego zbioru danych 
    \item Wykorzystanie algorytmu Gaussian Splatting do rekonstrukcji sceny 3D
    \item Filtracja chmury punktów przy użyciu różnych technik 
    \item Zastosowanie architektur sieci neuronowych takich jak PointNet do klasyfikacji chmury punktów 
    \item Adaptacja istniejących bibliotek do wizualizacji wyników 
    \item Implementacja własnego algotymu do renderowania gaussianów
\end{enumerate}