\section{Projekt}
\subsection{Specyfikacja wymagań na produkt programowy}
Czy my tu musimy dać diagramy wymagań? :O
\subsection{Architektura}
Projekt architektury, zastosowane wzorce projektowe???
\subsection{Implementacja}
Opis zastosowanych technologii i dokumentów

\subsubsection{Gaussian Splatting}
W celu uruchomienia algorytmu Gaussian Splatting korzystamy z gotowej implementacji którą 
zapewnia biblioteka \href{https://docs.gsplat.studio/main/index.html}{\textit{gsplat}} \cite{ye2024gsplatopensourcelibrarygaussian}. Znajdują się w niej metody które odzwierciedlają wykonywane kroki w teoretycznym opisie algorytmu (np. \textit{rasterization}) a także dodatkowe elementy oparte na nowszych badaniach które usprawniają proces trenowania i renderowania (np. \textit{sparse gradient}, \textit{absgrad}, itp.). Dodatkowym plusem biblioteki jest wsparcie do renderowania dużych scen. Jej użycie wymaga posiadania karty graficznej CUDA.  

Naszą kontrybucją jest dostosowanie istniejącego skryptu trenującego \href{https://github.com/nerfstudio-project/gsplat/blob/main/examples/simple_trainer.py}{simple\_trainer} tak aby działał on poprawnie i zgodnie z naszymi wymaganiami. Przyjmuje on następujące argumenty:

% change style to not highlight python specific words
\lstset{style=basicstyle}
\begin{lstlisting}[language=SHELXL] 
options:
-h, --help            show this help message and exit
--data_dir DATA_DIR   Path to the data directory.
--result_dir RESULT_DIR
                        Path to the results directory.
--data_factor DATA_FACTOR
                        Data factor.
--init_type {sfm,random}
                        Initialization type.
--strategy {default,mcmc}
                        Strategy type.
--max_steps MAX_STEPS
                        Maximum number of steps.
--init_num_pts INIT_NUM_PTS
                        Initial number of points (only for random).
--delta_steps DELTA_STEPS
                        Delta steps for evaluation and saving.
--scale_reg SCALE_REG
                        Scale regularization value.
--opacity_reg OPACITY_REG
                        Opacity regularization value.
--min_opacity MIN_OPACITY
                        Minimum opacity.
--refine_every REFINE_EVERY
                        Refine frequency (iterations).
--refine_start_iter REFINE_START_ITER
                        Refinement start iteration.
--reset_every RESET_EVERY
                        Reset opacities every this steps. [strategy=default]
--pause_refine_after_reset PAUSE_REFINE_AFTER_RESET
                        Pause refining GSs until this number of steps after reset. [strategy=default]
--cap_max CAP_MAX     Maximum cap for MCMC gaussians. [strategy=mcmc]
--sh_degree_interval SH_DEGREE_INTERVAL
                        Add spherical harmonics degree interval.
--init_scale INIT_SCALE
                        Initial scale.
--init_opa INIT_OPA   Initial opacity.
--packed PACKED       Use packed mode for rasterization.
--sparse_grad SPARSE_GRAD
                        Use sparse gradients for optimization.

\end{lstlisting}

\lstset{style=pythonstyle}

Z których obowiązkowe to \textit{--data\_dir} - ścieżka do folderu z rekonstrukcją i zdjęciami i \textit{--result\_dir} - ścieżka gdzie zostaną zapisane wyniki. 