\section{Podsumowanie}
\subsection{Osiągnięcie zamierzonych celów}

W ramach oceny projektu rozpatrzono osiągnięcie celów wyznaczonych na początku jego projektowania:

\begin{enumerate}
    \item skomponowanie własnego zbioru danych - przeprowadzono akwizycję danych na podstawie budynków Politechniki Wrocławskiej,
    \item wykorzystanie algorytmu Gaussian Splatting do rekonstrukcji sceny 3D - przystosowano skrypty implementujące algorytm Gaussian Splattingu do problemu dużych scen miejskich,
    \item filtracja chmury punktów przy użyciu różnych technik - w celu polepszenia osiągów modeli przed i po splattingu następują różne filtracje chmury punktów,
    \item zastosowanie architektur sieci neuronowych takich jak PointNet do klasyfikacji chmury punktów - wytrenowano i opisano model PointNet do segmentacji scen miejskich,
    \item adaptacja istniejących bibliotek do wizualizacji wyników - wdrożono alternatywne do własnego renderingu metody wizualizacji efektów Gaussian Splattingu i chmury punktów, bazujące na istniejących bibliotekach, 
    \item implementacja własnego algotymu do renderowania gaussianów - wytworzono oprogramowanie realizujące niskopoziomowy renderer chmury punktów i gaussianów, umożliwiający optymalną obliczeniowo wizualizację wyników obliczeń.
\end{enumerate}

Analiza potwierdza sprostanie założonym celom projektu.

\subsection{Wnioski}

Pozytywna ewaluacja osiągnięcia zamierzonych celów projektu pozwala uznać przedsięwzięcie za udane. Na szczególną uwagę zasługuje nieopisany, choć wynikający pośrednio z treści dokumentacji rozwój intelektualny w postaci poszerzenia, a w zasadzie zdobycia przez członków projektu wiedzy z obszaru pochodzącego z poza programu studiów inżynierskich, który dzięki niniejszemu projektowi był możliwy do osiągnięcia.

\subsection{Podziękowania}

Szczególne podziękowania należą się operatorce drona Paulinie, bez której z pewnością sukces w obszarze akwizycji danych nie byłby możliwy oraz opiekunowi pracy, dr hab. inż. Markowi Krótkiewiczowi, prof. ucz. oraz opiekunom pomocniczym, dr inż. Marcinowi Jodłowcowi, dr inż. Rafałowi Palakowi oraz dr inż. Zbigniewowi Telcowi.
