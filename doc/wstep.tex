\section{Wprowadzenie}

Przedmiotem projektu było wykonanie aplikacji wykorzystującej metody fotogrametrii do modelowania trójwymiarowych scen miejskich. Zaimplementowany program umożliwia użytkownikowi zrekonstruowanie chmury punktów i modelu 3D na podstawie podanych na wejściu zdjęć fotogrametrycznych za pomocą dostosowanych do specyfiki problemu w toku eksperymentów rozwiązań \textit{structure from motion} i \textit{gaussian splatting}. Otrzymana w ten sposób chmura może być przez użytkownika poddana segmentacji semantycznej, przeprowadzanej przez wytrenowany do tego celu model sztucznej inteligencji w postaci sieci neuronowej. Aplikacja oferuje również wizualizację wykonanych obliczeń, która możliwa jest dzięki użyciu przystosowanego dla większej wydajności mechanizmu renderowania.

Wytworzony produkt informatyczny ze względu na integrację wielu rozwiązań i efektywną implementację procesu \textit{end-to-end} przejawia potencjał w zastosowaniach biznesowych począwszy od branż takich jak gry wideo, przez architekturę, robotykę, pojazdy autonomiczne, skończywszy na modelowaniu urbanistycznym.

\subsection{Cel i zakres prac}

Rekonstrukcja, klasyfikacja i wizualizacja scen urbanistycznych to dynamicznie rozwijające się zagadnienie, które zyskało na znaczeniu dzięki rosnącej dostępności nowoczesnych technologii, takich jak LiDAR, oraz postępowi w dziedzinie sztucznej inteligencji. Kluczowym wyzwaniem pozostaje jednak efektywne przetwarzanie ogromnych zbiorów danych – chmur punktów, które nierzadko obejmują miliony elementów. Choć na rynku istnieją liczne programy i algorytmy wspierające tego typu analizy, ich skuteczne wykorzystanie w praktyce bywa problematyczne, głównie z uwagi na skalę i złożoność danych urbanistycznych.

Rozwiązanie ma na celu umożliwienie przeprowadzenia rekonstrukcji do modelu trójwymiarowego na podstawie odpowiednio przygotowanego zbioru zdjęć, klasyfikacji otrzymanej sceny na zbiór pre-definiowanych klas istotnych w kontekście urbanistycznym oraz wizualizacji wykonanych obliczeń. 

Poniżej znajdują się cele, które miały zostać zrealizowane w przedsięwzięciu:

\begin{enumerate}
    \item skomponowanie własnego zbioru danych, 
    \item wykorzystanie algorytmu Gaussian Splatting do rekonstrukcji sceny 3D,
    \item filtracja chmury punktów przy użyciu różnych technik, 
    \item zastosowanie architektur sieci neuronowych takich jak PointNet do klasyfikacji chmury punktów, 
    \item adaptacja istniejących bibliotek do wizualizacji wyników, 
    \item implementacja własnego algorytmu do renderowania.
\end{enumerate}

\subsection{Wyzwania i ograniczenia}

Realizacja projektu wiązała się z analizą szeregu wyzwań i ograniczeń, między innymi poniżej przedstawionych kwestii.

\begin{enumerate}
    \item \textbf{Przetwarzanie dużych zbiorów danych}   
    
    Aby sprostać zadaniu przetwarzania setek zdjęć czy milionów punktów, najbardziej kosztowne obliczenia tj. trenowanie modelu podczas Gaussian Splattingu zostały z wykorzystaniem technologii CUDA przeniesione na GPU, a w przypadku segmentacji wykorzystano technikę \textit{chunkowania}. Aplikacja wymaga od użytkownika posiadania systemu z kompatybilnym sprzętem GPU i obsługą CUDA.
    \item \textbf{Niezbalansowana reprezentacja kategorii semantycznych}    
    
    Wybrane podejścia mające na celu poprawę jakości segmentacji zostały opisane w późniejszych sekcjach dokumentacji.
    \item \textbf{Ograniczona dostępność danych}    
    
    Za część prac przyjęto przeprowadzenie akwizycji danych i skompletowanie własnych zbiorów danych odpowiadających potrzebom projektu. 
\end{enumerate}