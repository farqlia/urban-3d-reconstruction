% Document class and language
\documentclass{article}

\usepackage{mib}

% inspiration: https://www.overleaf.com/project/673f933029a56f1eca2314c4

\usepackage{lmodern}
\usepackage[T2A]{fontenc}
\usepackage{color}
\usepackage[utf8]{inputenc} % allow utf-8 input
\usepackage[T1]{fontenc}    % use 8-bit T1 fonts
\usepackage{hyperref}       % hyperlinks
\usepackage{url}            % simple URL typesetting
\usepackage{booktabs}       % professional-quality tables
\usepackage{amsfonts}       % blackboard math symbols
\usepackage{nicefrac}       % compact symbols for 1/2, etc.
\usepackage{microtype}      % microtypography
\usepackage{lipsum}
\usepackage{graphicx}
\usepackage{color}
\usepackage{amsthm,amsmath,amssymb}
\usepackage[english, polski]{babel}
\usepackage{biblatex}

% \usepackage[a4paper,top=2cm,bottom=2cm,left=3cm,right=3cm,marginparwidth=1.75cm]{geometry}

% Useful packages
\usepackage{csvsimple}
\usepackage{graphicx}
\usepackage{subcaption}
\usepackage{forest}
\usepackage{listings}
%\PassOptionsToPackage{silent}{fontspec}
%\graphicspath{ {./images/} }
\setcounter{page}{1}
% \addbibresource{literature.bib}

\title{Metoda trójwymiarowego modelowania obszarów urbanistycznych z wykorzystaniem metod fotogrametrii}

\author{
    Daniel Borkowski\\
    \And
    Julia Farganus\\
    \And
    Rafał Mielniczuk\\
    \And
    Katarzyna Wochal\\
    \And 
    Opiekun pracy\\
    dr hab. inż. Marek Krótkiewicz, prof. PWr \\
    \small Politechnika Wrocławska\\
    \small Wydział Informatyki i Telekomunikacji\\
}

\date{12.2024}

\begin{document}

\maketitle

\begin{titlepage}

    \newcommand\concat[3]{\left[#1 \parallel_#3 #2\right]}

    \begin{center}

        \huge
        \textbf{Metoda trójwymiarowego modelowania obszarów urbanistycznych \\z wykorzystaniem metod fotogrametrii}
        \vspace*{1cm}

        \author{
            \small Daniel Borkowski\\
            \and
            \small Julia Farganus\\
            \and
            \small Rafał Mielniczuk\\
            \and
            \small Katarzyna Wochal\\
        }

        \small
        \vspace{1.5cm}
        Opiekun pracy\\
        dr hab. inż. Marek Krótkiewicz, prof. PWr
        \vspace{1.5cm}

        Politechnika Wrocławska\\
        Wydział Informatyki i Telekomunikacji\\
        \vspace*{1cm}

        \begin{abstract}
            Celem pracy jest wykonanie aplikacji, która wykorzystuje metody fotogrametrii do modelowania miejskich scen 3D. Dane wejściowe stanowią zdjęcia obszarów miejskich, które są przetwarzane w celu stworzenia modelu 3D, a następnie segmentowane na obiekty przestrzeni miejskiej, takie jak budynki, tereny zielone, itp.. Aplikacja będzie wizualizować model oraz wyniki segmentacji semantycznej.

            Innowacyjność tego projektu polega na połączeniu, adaptacji i udoskonaleniu najlepszych dostępnych rozwiązań, takich jak Gaussian Splatting i PointNet, aby stworzyć nowy, kompleksowy produkt.

            Przetwarzanie dużych scen miejskich jest wyzwaniem dla obecnie istniejących rozwiązań, które skupiają się głównie na pojedynczych obiektach lub zamkniętych scenach. Typowa scena miejska natomiast może obejmować setki zdjęć, a powstała chmura może zawierać miliony punktów. Dodatkowym wyzwaniem jest niezbalansowana reprezentacja kategorii semantycznych. Nasze rozwiązanie ma na celu efektywne przetwarzanie dużych zbiorów danych przy rozsądnym zużyciu zasobów czasowych i pamięciowych.

            Zastosowania biznesowe otrzymywanych w ten sposób modeli 3D są szerokie: od gier wideo, przez architekturę, robotykę, pojazdy autonomiczne, po modelowanie urbanistyczne.
        \end{abstract}

        \vfill
            
            
        \vspace{0.8cm}
            
            
        \Large
        Wrocław, 2024
        
            
    \end{center}
\end{titlepage}

\section{Zarys projektu}

\section{Rozwinięcie}

\subsection{Wstęp}


\subsection{Stan nauki}


\subsection{Wyniki}

\section{Podsumowanie}

\subsection{Wnioski}

\subsection{Kierunki rozwoju}

\subsection{Podziękowania}
Dziekujemy operatorowi drona i wojciechowi cebuli




% \printbibliography

\end{document}